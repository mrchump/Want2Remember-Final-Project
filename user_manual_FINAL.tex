\documentclass[12pt]{article}
\usepackage[margin=1in]{geometry}
\usepackage{hyperref}
\usepackage{titlesec}
\usepackage{fancyhdr}
\usepackage{titling}
\usepackage{lmodern}
\usepackage{setspace}

\pagestyle{fancy}
\fancyhf{}
\rhead{Want2Remember User Manual}
\lhead{Kevin Bayona-Galindo \\ Nikolazi Tartinsky}
\cfoot{\thepage}

\titleformat{\section}[block]{\normalfont\Large\bfseries}{\thesection}{1em}{}

\begin{document}

% Standalone centered title page
\begin{titlepage}
    \newcommand{\HRule}{\rule{\linewidth}{0.5mm}} 
    \vspace*{\fill}
    \begin{center}
        \HRule \\[0.5cm]
        {\Huge \bfseries User Manual \\[0.4cm]}
        \HRule \\[1.5cm]
        {\LARGE \textbf{Want2Remember Project}}\\[0.5cm]
        {\Large Final Documentation – May 6, 2025}\\[2cm]
        {\Large Kevin Bayona-Galindo \& Nikolazi Tartinsky}\\[0.5cm]
        {\large \today}
    \end{center}
    \vspace*{\fill}
\end{titlepage}

\tableofcontents
\newpage

\section{Introduction}
Want2Remember is a web-based application designed to help users store, view, and delete short memory entries. Built using HTML, CSS, and JavaScript, it persists data via the browser’s \texttt{localStorage}. The app is styled for clarity and accessibility and runs either locally or within a Docker container.

\section{Project Objectives}
\begin{itemize}
  \item Build a memory tracking app for cognitive and practical memory support
  \item Implement key functions: add, view, and delete memories
  \item Design a clean and usable interface
  \item Preserve data using local browser storage
\end{itemize}

\section{How to Run the Application}
\begin{enumerate}
  \item Clone the repository from GitHub:
    \begin{verbatim}
    git clone https://github.com/kbthepioneer/Want2Remember-Final-Project.git
    \end{verbatim}
  \item Open the \texttt{index.html} file in any web browser.
  \item Alternatively, use a local development server or Docker setup.
\end{enumerate}

\section{How to Use the App}
\begin{enumerate}
  \item Enter a memory title and note in the input form.
  \item Click \textbf{Save Memory} to add it to the memory list.
  \item Each memory can be deleted by clicking the \textbf{X} button.
  \item The app automatically saves all data using \texttt{localStorage}.
\end{enumerate}

\section{Troubleshooting \& Notes}
\begin{itemize}
  \item Make sure JavaScript is enabled in your browser.
  \item For a clean refresh, clear browser localStorage and reload.
  \item The app is designed for modern desktop and mobile browsers.
\end{itemize}

\section{Final Updates and Submission Summary}
\begin{itemize}
  \item Converted all snapshot objective documents to LaTeX and added standalone title pages.
  \item Cleaned and finalized all SDD, SRS, design spec, and user manual documents.
  \item Removed outdated \texttt{.docx} and older \texttt{.tex} versions.
  \item Updated GitHub repository structure and confirmed successful push of all deliverables.
\end{itemize}

\end{document}
