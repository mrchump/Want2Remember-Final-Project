\documentclass[12pt]{article}
\usepackage[margin=1in]{geometry}
\usepackage{longtable}
\usepackage{hyperref}
\usepackage{titlesec}
\usepackage{fancyhdr}
\usepackage{titling}
\usepackage{lmodern}
\usepackage{setspace}

\pagestyle{fancy}
\fancyhf{}
\rhead{Want2Remember SRS}
\lhead{Kevin Bayona-Galindo \\ Nikolazi Tartinsky}
\cfoot{\thepage}

\titleformat{\section}[block]{\normalfont\Large\bfseries}{\thesection}{1em}{}

\begin{document}

% Centered standalone title page
\begin{titlepage}
    \newcommand{\HRule}{\rule{\linewidth}{0.5mm}} 
    \vspace*{\fill}
    \begin{center}
        \HRule \\[0.5cm]
        {\Huge \bfseries Software Requirements Specification \\[0.4cm]}
        \HRule \\[1.5cm]
        {\LARGE \textbf{Want2Remember Project}}\\[0.5cm]
        {\Large Final Documentation – May 6, 2025}\\[2cm]
        {\Large Kevin Bayona-Galindo \& Nikolazi Tartinsky}\\[0.5cm]
        {\large \today}
    \end{center}
    \vspace*{\fill}
\end{titlepage}

\tableofcontents
\newpage

\section*{Version History}
\begin{longtable}{|p{3cm}|p{3cm}|p{6cm}|p{4cm}|}
\hline
\textbf{Version} & \textbf{Date} & \textbf{Description} & \textbf{Authors} \\
\hline
v1.0 & 04/15/2025 & Initial requirement draft & Kevin \& Niko \\
\hline
v2.0 & 04/22/2025 & Added persistence and validation logic & Kevin \& Niko \\
\hline
v3.0 & 04/29/2025 & Included delete features and UI polish & Kevin \& Niko \\
\hline
v4.0 & 05/05/2025 & Final LaTeX formatting + requirement check & Kevin \& Niko \\
\hline
v5.0 & 05/07/2025 & Final SRS enhancements and structure updates & Kevin \& Niko \\
\hline
V.60 & 05/09/2025 & Included a new feature for the product & Donald \\
\hline
\end{longtable}

\section{Introduction}
\subsection*{Purpose}
This document defines the functional, non-functional, and system requirements for Want2Remember, a single-page web application to help users store personal memories.

\subsection*{Intended Audience}
This document is intended for the course instructor, student developers, and evaluators reviewing the software requirements of the final project.

\subsection*{Overview}
Want2Remember enables users to add, view, and delete memory entries. Data persists using browser-based `localStorage`. The app is styled for accessibility and runs locally or in a Docker container.

\section{Overall Description}
\subsection*{Product Perspective}
Want2Remember is a standalone single-page web application. It does not require login or backend services and is designed for quick deployment and use on any modern browser.

\subsection*{Product Functions}
\begin{itemize}
    \item Add new memory entries (title + note)
    \item Display all saved memories
    \item Search for all saved memories
    \item Delete selected memories
    \item Automatically persist data via localStorage
\end{itemize}

\subsection*{User Characteristics}
The typical user is a student or casual user who wants to track reminders or personal notes. No technical knowledge is required.

\subsection*{Assumptions and Dependencies}
\begin{itemize}
    \item JavaScript and localStorage must be enabled in the user’s browser
    \item No internet connection required after app load
    \item Runs best in Chrome, Firefox, or modern mobile browsers
\end{itemize}

\section{External Interface Requirements}
\subsection*{User Interfaces}
The UI includes:
\begin{itemize}
    \item A memory input form (title + note)
    \item A list of stored memories styled as cards
    \item Delete button next to each memory
    \item Responsive layout and dark-themed styling
\end{itemize}

\subsection*{Software Interfaces}
\begin{itemize}
    \item JavaScript DOM interaction
    \item Browser `localStorage` API
    \item Runs locally in browser or via Docker
\end{itemize}

\section{Functional Requirements}
\begin{itemize}
    \item Users can enter memory title and note
    \item Users can submit a memory using a form
    \item App stores the memory in `localStorage`
    \item App displays all saved memories in a styled list
    \item Users can delete memories with a ❌ button
    \item Deleted memories do not reappear on refresh
\end{itemize}

\section{Non-functional Requirements}
\begin{itemize}
    \item The app must function correctly in all modern browsers.
    \item The UI should be responsive and accessible.
    \item The app should not require internet access after initial loading.
    \item Page reloads must not erase saved data.
    \item Search allows user to search saved memories
\end{itemize}

\section{Legal and Ethical Considerations}
\begin{itemize}
    \item No user accounts or private data stored
    \item All data is saved only in the user's browser
    \item Future versions may require data encryption if cloud sync is implemented
    \item Accessibility: UI designed with simplicity and readability in mind
\end{itemize}

\section{Test Plan Summary}
\begin{itemize}
    \item Validate memory is saved on form submission
    \item Confirm deletion removes memory from UI and `localStorage`
    \item Verify persistence across browser reloads
    \item Verify if search bar searches for memories
    \item Check UI responsiveness on mobile and desktop
\end{itemize}

\section{Glossary}
\begin{itemize}
    \item CRUD – Create, Read, Update, Delete
    \item UI – User Interface
    \item API – Application Programming Interface
    \item DOM – Document Object Model
    \item localStorage – Key-value browser storage for user data
\end{itemize}

\section{References}
\begin{itemize}
  \item \url{https://developer.mozilla.org/en-US/docs/Web/API/Window/localStorage}
  \item \url{https://hub.docker.com/_/nginx}
  \item \url{https://www.docker.com/products/docker-desktop}
  \item \url{https://overleaf.com}
  \item \url{https://www.atlassian.com/software/jira}
  \item \url{https://github.com/kbthepioneer/Want2Remember-Final-Project}
\end{itemize}

\section{Conclusion}
This Software Requirements Specification (SRS) defines the intended functionality and behavior of the Want2Remember application. It outlines a lightweight, accessible tool for storing memory entries on the client side using browser technologies. The document provides a foundation for ongoing development and future enhancements.

\end{document}
